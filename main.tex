% Created 2024-04-03 Wed 14:28
% Intended LaTeX compiler: pdflatex
\documentclass[energies,article,submit,moreauthors]{Definitions/mdpi}

%--------------------
% Class Options:
%--------------------
%----------
% journal
%----------
% Choose between the following MDPI journals:
% acoustics, actuators, addictions, admsci, adolescents, aerobiology, aerospace, agriculture, agriengineering, agrochemicals, agronomy, ai, air, algorithms, allergies, alloys, analytica, analytics, anatomia, animals, antibiotics, antibodies, antioxidants, applbiosci, appliedchem, appliedmath, applmech, applmicrobiol, applnano, applsci, aquacj, architecture, arm, arthropoda, arts, asc, asi, astronomy, atmosphere, atoms, audiolres, automation, axioms, bacteria, batteries, bdcc, behavsci, beverages, biochem, bioengineering, biologics, biology, biomass, biomechanics, biomed, biomedicines, biomedinformatics, biomimetics, biomolecules, biophysica, biosensors, biotech, birds, bloods, blsf, brainsci, breath, buildings, businesses, cancers, carbon, cardiogenetics, catalysts, cells, ceramics, challenges, chemengineering, chemistry, chemosensors, chemproc, children, chips, cimb, civileng, cleantechnol, climate, clinpract, clockssleep, cmd, coasts, coatings, colloids, colorants, commodities, compounds, computation, computers, condensedmatter, conservation, constrmater, cosmetics, covid, crops, cryptography, crystals, csmf, ctn, curroncol, cyber, dairy, data, ddc, dentistry, dermato, dermatopathology, designs, devices, diabetology, diagnostics, dietetics, digital, disabilities, diseases, diversity, dna, drones, dynamics, earth, ebj, ecologies, econometrics, economies, education, ejihpe, electricity, electrochem, electronicmat, electronics, encyclopedia, endocrines, energies, eng, engproc, entomology, entropy, environments, environsciproc, epidemiologia, epigenomes, est, fermentation, fibers, fintech, fire, fishes, fluids, foods, forecasting, forensicsci, forests, foundations, fractalfract, fuels, future, futureinternet, futurepharmacol, futurephys, futuretransp, galaxies, games, gases, gastroent, gastrointestdisord, gels, genealogy, genes, geographies, geohazards, geomatics, geosciences, geotechnics, geriatrics, grasses, gucdd, hazardousmatters, healthcare, hearts, hemato, hematolrep, heritage, higheredu, highthroughput, histories, horticulturae, hospitals, humanities, humans, hydrobiology, hydrogen, hydrology, hygiene, idr, ijerph, ijfs, ijgi, ijms, ijns, ijpb, ijtm, ijtpp, ime, immuno, informatics, information, infrastructures, inorganics, insects, instruments, inventions, iot, j, jal, jcdd, jcm, jcp, jcs, jcto, jdb, jeta, jfb, jfmk, jimaging, jintelligence, jlpea, jmmp, jmp, jmse, jne, jnt, jof, joitmc, jor, journalmedia, jox, jpm, jrfm, jsan, jtaer, jvd, jzbg, kidneydial, kinasesphosphatases, knowledge, land, languages, laws, life, liquids, literature, livers, logics, logistics, lubricants, lymphatics, machines, macromol, magnetism, magnetochemistry, make, marinedrugs, materials, materproc, mathematics, mca, measurements, medicina, medicines, medsci, membranes, merits, metabolites, metals, meteorology, methane, metrology, micro, microarrays, microbiolres, micromachines, microorganisms, microplastics, minerals, mining, modelling, molbank, molecules, mps, msf, mti, muscles, nanoenergyadv, nanomanufacturing,\gdef\@continuouspages{yes}} nanomaterials, ncrna, ndt, network, neuroglia, neurolint, neurosci, nitrogen, notspecified, %%nri, nursrep, nutraceuticals, nutrients, obesities, oceans, ohbm, onco, %oncopathology, optics, oral, organics, organoids, osteology, oxygen, parasites, parasitologia, particles, pathogens, pathophysiology, pediatrrep, pharmaceuticals, pharmaceutics, pharmacoepidemiology,\gdef\@ISSN{2813-0618}\gdef\@continuous pharmacy, philosophies, photochem, photonics, phycology, physchem, physics, physiologia, plants, plasma, platforms, pollutants, polymers, polysaccharides, poultry, powders, preprints, proceedings, processes, prosthesis, proteomes, psf, psych, psychiatryint, psychoactives, publications, quantumrep, quaternary, qubs, radiation, reactions, receptors, recycling, regeneration, religions, remotesensing, reports, reprodmed, resources, rheumato, risks, robotics, ruminants, safety, sci, scipharm, sclerosis, seeds, sensors, separations, sexes, signals, sinusitis, skins, smartcities, sna, societies, socsci, software, soilsystems, solar, solids, spectroscj, sports, standards, stats, std, stresses, surfaces, surgeries, suschem, sustainability, symmetry, synbio, systems, targets, taxonomy, technologies, telecom, test, textiles, thalassrep, thermo, tomography, tourismhosp, toxics, toxins, transplantology, transportation, traumacare, traumas, tropicalmed, universe, urbansci, uro, vaccines, vehicles, venereology, vetsci, vibration, virtualworlds, viruses, vision, waste, water, wem, wevj, wind, women, world, youth, zoonoticdis
% For posting an early version of this manuscript as a preprint, you may use "preprints" as the journal. Changing "submit" to "accept" before posting will remove line numbers.

%---------
% article
%---------
% The default type of manuscript is "article", but can be replaced by:
% abstract, addendum, article, book, bookreview, briefreport, casereport, comment, commentary, communication, conferenceproceedings, correction, conferencereport, entry, expressionofconcern, extendedabstract, datadescriptor, editorial, essay, erratum, hypothesis, interestingimage, obituary, opinion, projectreport, reply, retraction, review, perspective, protocol, shortnote, studyprotocol, systematicreview, supfile, technicalnote, viewpoint, guidelines, registeredreport, tutorial
% supfile = supplementary materials

%----------
% submit
%----------
% The class option "submit" will be changed to "accept" by the Editorial Office when the paper is accepted. This will only make changes to the frontpage (e.g., the logo of the journal will get visible), the headings, and the copyright information. Also, line numbering will be removed. Journal info and pagination for accepted papers will also be assigned by the Editorial Office.

%------------------
% moreauthors
%------------------
% If there is only one author the class option oneauthor should be used. Otherwise use the class option moreauthors.

%---------
% pdftex
%---------
% The option pdftex is for use with pdfLaTeX. Remove "pdftex" for (1) compiling with LaTeX & dvi2pdf (if eps figures are used) or for (2) compiling with XeLaTeX.

%=================================================================
% MDPI internal commands - do not modify
\firstpage{1}
\makeatletter
\setcounter{page}{\@firstpage}
\makeatother
\pubvolume{1}
\issuenum{1}
\articlenumber{0}
\pubyear{2024}
\copyrightyear{2024}
%\externaleditor{Academic Editor: Firstname Lastname}
\datereceived{ }
\daterevised{ } % Comment out if no revised date
\dateaccepted{ }
\datepublished{ }
%\datecorrected{} % For corrected papers: "Corrected: XXX" date in the original paper.
%\dateretracted{} % For corrected papers: "Retracted: XXX" date in the original paper.
\hreflink{https://doi.org/} % If needed use \linebreak
%\doinum{}
%\pdfoutput=1 % Uncommented for upload to arXiv.org
%\CorrStatement{yes}  % For updates


%=================================================================
% Add packages and commands here. The following packages are loaded in our class file: fontenc, inputenc, calc, indentfirst, fancyhdr, graphicx, epstopdf, lastpage, ifthen, float, amsmath, amssymb, lineno, setspace, enumitem, mathpazo, booktabs, titlesec, etoolbox, tabto, xcolor, colortbl, soul, multirow, microtype, tikz, totcount, changepage, attrib, upgreek, array, tabularx, pbox, ragged2e, tocloft, marginnote, marginfix, enotez, amsthm, natbib, hyperref, cleveref, scrextend, url, geometry, newfloat, caption, draftwatermark, seqsplit
% cleveref: load \crefname definitions after \begin{document}

%=================================================================
% Please use the following mathematics environments: Theorem, Lemma, Corollary, Proposition, Characterization, Property, Problem, Example, ExamplesandDefinitions, Hypothesis, Remark, Definition, Notation, Assumption
%% For proofs, please use the proof environment (the amsthm package is loaded by the MDPI class).

%=================================================================
% Full title of the paper (Capitalized)
\Title{A Simulated Annealing Approach to the Scheduling of Battery-Electric Bus Charging}

% MDPI internal command: Title for citation in the left column
\TitleCitation{A Simulated Annealing Approach to the Scheduling of Battery-Electric Bus Charging}

% Author Orchid ID: enter ID or remove command
%\newcommand{\orcidauthorA}{0000-0000-0000-000X} % Add \orcidA{} behind the author's name
%\newcommand{\orcidauthorB}{0000-0000-0000-000X} % Add \orcidB{} behind the author's name

% Authors, for the paper (add full first names)
\Author{Alexander Brown$^{1,\dagger,}$* and Greg Droge$^{2,\dagger,\ddagger}$}

%\longauthorlist{yes}

% MDPI internal command: Authors, for metadata in PDF
\AuthorNames{Alexander Brown and Greg Droge}

% MDPI internal command: Authors, for citation in the left column
\AuthorCitation{Brown, A.; Droge G.}
% If this is a Chicago style journal: Lastname, Firstname, Firstname Lastname, and Firstname Lastname.

% Affiliations / Addresses (Add [1] after \address if there is only one affiliation.)
\address{%
$^{1}$ \quad Utah State University; a01704744@usu.edu\\
$^{2}$ \quad Utah State University; greg.droge@usu.edu}

% Contact information of the corresponding author
\corres{Correspondence: a01704744@usu.edu}

% Current address and/or shared authorship
\firstnote{Current address: 102 Old Main Hl, Logan, UT 84322-0102.}  % Current address should not be the same as any items in the Affiliation section.
\secondnote{These authors contributed equally to this work.}
% The commands \thirdnote{} till \eighthnote{} are available for further notes

%\simplesumm{} % Simple summary

%\conference{} % An extended version of a conference paper

% The fields PACS, MSC, and JEL may be left empty or commented out if not applicable
%\PACS{J0101}
%\MSC{}
%\JEL{}

%%%%%%%%%%%%%%%%%%%%%%%%%%%%%%%%%%%%%%%%%%
% Only for the journal Diversity
%\LSID{\url{http://}}

%%%%%%%%%%%%%%%%%%%%%%%%%%%%%%%%%%%%%%%%%%
% Only for the journal Applied Sciences
%\featuredapplication{Authors are encouraged to provide a concise description of the specific application or a potential application of the work. This section is not mandatory.}
%%%%%%%%%%%%%%%%%%%%%%%%%%%%%%%%%%%%%%%%%%

%%%%%%%%%%%%%%%%%%%%%%%%%%%%%%%%%%%%%%%%%%
% Only for the journal Data
%\dataset{DOI number or link to the deposited data set if the data set is published separately. If the data set shall be published as a supplement to this paper, this field will be filled by the journal editors. In this case, please submit the data set as a supplement.}
%\datasetlicense{License under which the data set is made available (CC0, CC-BY, CC-BY-SA, CC-BY-NC, etc.)}

%%%%%%%%%%%%%%%%%%%%%%%%%%%%%%%%%%%%%%%%%%
% Only for the journal Toxins
%\keycontribution{The breakthroughs or highlights of the manuscript. Authors can write one or two sentences to describe the most important part of the paper.}

%%%%%%%%%%%%%%%%%%%%%%%%%%%%%%%%%%%%%%%%%%
% Only for the journal Encyclopedia
%\encyclopediadef{For entry manuscripts only: please provide a brief overview of the entry title instead of an abstract.}

%%%%%%%%%%%%%%%%%%%%%%%%%%%%%%%%%%%%%%%%%%
% Only for the journal Advances in Respiratory Medicine
%\addhighlights{yes}
%\renewcommand{\addhighlights}{%

%\noindent This is an obligatory section in “Advances in Respiratory Medicine”, whose goal is to increase the discoverability and readability of the article via search engines and other scholars. Highlights should not be a copy of the abstract, but a simple text allowing the reader to quickly and simplified find out what the article is about and what can be cited from it. Each of these parts should be devoted up to 2~bullet points.\vspace{3pt}\\
%\textbf{What are the main findings?}
% \begin{itemize}[labelsep=2.5mm,topsep=-3pt]
% \item First bullet.
% \item Second bullet.
% \end{itemize}\vspace{3pt}
%\textbf{What is the implication of the main finding?}
% \begin{itemize}[labelsep=2.5mm,topsep=-3pt]
% \item First bullet.
% \item Second bullet.
% \end{itemize}
%}

\usepackage{subcaption}                     % Subfigures
\usepackage[ruled]{algorithm2e}             % Algorithms
\usepackage{listings}                       % Code in LaTeX
\usepackage{listings-rust}                  % Code in LaTeX
\usepackage{amsfonts}                       % Cool math fonts
\usepackage{tabularx}                       % Cool tables
\usepackage{multicol}                       % Add capability to make columns
\usepackage{soul}                           % Highlight text
\usepackage[stable]{footmisc}               % Allow footnotes in section headers
\setlength\parindent{0pt}                   % No indent for paragraphs
\usepackage{xfp}                             % No trailing zeros
\lstset{language=Rust, style=boxed}
\usetikzlibrary{arrows.meta}                % Arrows for tikz
\newcommand{\Or}{\textbf{ or }}
\renewcommand*{\And}{\textbf{ and }}
\newcolumntype{L}[1]{>{\hsize=#1\hsize\raggedright\arraybackslash}X}%
\newcommand{\mathcolorbox}[1]{\colorbox{yellow}{$\displaystyle #1$}}
\newcommand\mycommfont[1]{\footnotesize\ttfamily\textcolor{gray}{#1}}
\newcommand{\T}{\mathcal{T}}                % To make it clear the difference
\newcommand{\Tau}{T}                        % between Tau and T
\newcommand{\AC}{AC(u, d, v, \eta)}         % Set the parameters for AC once
\newcommand{\UC}{UC(u, d, v)}               % Set the parameters for UC once
\newcommand{\ACi}{AC(u_i, d_i, q_i, \eta_i)}% Set the parameters for AC once
\newcommand{\UCi}{UC(u_i, d_i, q_i)}        % Set the parameters for UC once
\newcommand{\Not}{\textbf{not }}            % Custom `not' operator
\newcommand{\visit}{(i, b, a, e, u, d, q, \eta, \xi)}
\newcommand{\I}{\mathbb{I}}                 % Set of visit tuples
\newcommand{\C}{\mathbb{C}}                 % Charger availability information
\newcommand{\U}{\mathcal{U}}                % Uniform distribution
\newcommand{\W}{\mathcal{W}}                % Weighted distribution
\newcommand{\Sol}{\mathbb{S}}               % A shorthand for visit tuple
\newcommand{\M}{\mathbb{M}}                 % A shorthand for the metadata
\newcommand{\Hd}{\mathbb{H}}                % Set of discrete times
\newcommand{\Nu}{\mathcal{V}}               % Draw a nice Nu
\newcommand{\Iset}{I}                       % Set of visits 1-I
\newcommand{\Isetinit}{I_0}                 % Set of visits inital visits
\newcommand{\Isetfinal}{I_f}                % Set of visits final visits
\newcommand{\Bset}{B}                       % Set of visits 1-B
\newcommand{\Qset}{Q}                       % Set of visits 1-Q
\newcommand{\Jset}{J}                       % Set of visits 1-J
\newcommand{\Jsetq}{\mathbb{J}}             % Set of visits 1-J for queue active times
\newcommand{\Hset}{H}                       % Set of visits 1-H
%%-------------------------------------------------------------------------------
% Experiment parameters
\newcommand{\A}{35 }                                                            % Number of buses
\newcommand{\N}{338 }                                                           % Number of visits
\newcommand{\Cgain}{5000}                                                       % Gain applied to penalty method
\newcommand{\acharge}{0.9}                                                      % BOD charge percentage
\newcommand{\bcharge}{0.7 }                                                     % EOD charge percentage
\newcommand{\mincharge}{25\% }                                                  % Min visit charge percent
\newcommand{\minchargeD}{0.25 }                                                 % Min visit charge decimal
\newcommand{\maxcharge}{100\% }                                                 % Max visit charge percent
\newcommand{\batsize}{388 }                                                     % Battery capacity
\newcommand{\fast}{15 }                                                         % Number of fast chargers
\newcommand{\slow}{15 }                                                         % Number of slow chargers
\newcommand{\fasts}{911 }                                                       % Speed of fast charger
\newcommand{\slows}{30 }                                                        % Speed of slow charger
\newcommand{\contvars}{7,511 }
\newcommand{\intvars}{328,282 }
\newcommand{\localcnt}{500 }                                                    % Number of local search iterations
\newcommand{\tempinit}{9000 }                                                   % Initial temperature
\newcommand{\tempcnt}{9101 }                                                    % Number of steps in temperature
\newcommand{\quicklocal}{0.25 }                                                % Time to finish local quick
\newcommand{\heuristiclocal}{0.4 }                                             % Time to finish local heuristic
%%-------------------------------------------------------------------------------
%% Solve output
%% Solve output
\newcommand{\timeran}{4.2 }                                                    % Time ran for MILP [s]
% Abstract (Do not insert blank lines, i.e. \\)
\abstract{With an increasing adoption of Battery Electric Bus (BEB) fleets, developing a reliable charging schedule is vital to a successful migration from their fossil fuel counterparts. In this paper, a Simulated Annealing (SA) implementation for a charge scheduling framework for a fixed-schedule fleet of BEBs that utilizes a proportional battery dynamics model, accounts for multiple charger types, allows partial charging, and further considers the total energy consumed by the schedule as well as peak power use is developed. Two generation mechanisms are implemented for the SA algorithm denoted as the "quick" and "heuristic" implementations, respectively. The model validity is demonstrated by utilizing a set of routes sampled form the Utah Transit Authority (UTA) and comparing the results two other models: the MILP PAP and the Qin-Modified. The results presented show that the "heuristic" approach was able to generate a solution comparable to that of the MILP PAP over similar execution times.}

% Keywords
\keyword{Battery Electric Bus (BEB), Charge Scheduling, Simulated Annealing, Position Allocation Problem (PAP), Mixed Integer Linear Program (MILP)}
\author{Alexander Brown}
\date{\today}
\title{A Simulated Annealing Approach to the Scheduling of Battery-Electric Bus Charging}
\begin{document}

\maketitle
\tableofcontents

\parskip 3mm                                % Set the vetical space between paragraphs
\let\ref\autoref                            % Redifine `\ref` as `\autoref` because lazy
\SetCommentSty{mycommfont}                  % Set the comment color

\section{Introduction}
\label{sec:sa-introduction}
Public transportation systems are a critical component urban areas. An increased awareness and concern of environmental
impacts of petroleum based public transportation has driven an effort to reduce the pollutant footprint
\cite{de-2014-simul-elect,xylia-2018-role-charg,guida-2017-zeeus-repor-europ,li-2016-batter-elect}. Particularly,
the electrification of public bus transportation via battery power, i.e., battery electric buses (BEBs), has received
significant attention \cite{li-2016-batter-elect}. Although the technology provides benefits beyond reduction in
emissions, such as lower driving costs, lower maintenance costs, and reduced vehicle noise, battery powered systems
introduce new challenges such as larger upfront costs, and potentially several hours long ``refueling'' periods
\cite{xylia-2018-role-charg,li-2016-batter-elect}. Furthermore, the problem is exacerbated by the constraints of the
transit schedule to which the fleet must adhere, the limited amount of chargers available, and the adverse affects in
the health of the battery due to fast charging \cite{lutsey-2019-updat-elect}. This work presents a scheduling
framework for a fixed-schedule fleet of BEBs that utilizes a proportional battery dynamics model, accounts for multiple
charger types, allows partial charging, and further considers the total energy consumed by the schedule as well as peak
power use.

Literature shows an interest in solving the problem of assigning BEBs to charging queues or optimizing their
infrastructure \cite{wei-2018-optim-spatio,sebastiani-2016-evaluat-elect,hoke-2014-accoun-lithium,wang-2017-elect-vehic}. Additionally, the prospect of solving both problems simultaneously
has received much attention \cite{wei-2018-optim-spatio,sebastiani-2016-evaluat-elect,hoke-2014-accoun-lithium,wang-2017-elect-vehic}. These problems vary by including assignment of buses to routes
\cite{rinaldi-2020-mixed-fleet,zhou-2020-collab-optim,tang-2019-robus-sched,li-2014-trans-bus}, determining
whether a set of existing combustion based buses should be replaced with BEBs \cite{zhou-2020-bi-objec,duan-2021-refor-mixed,rinaldi-2020-mixed-fleet,zhou-2020-collab-optim}, and accounting for uncertainties
\cite{bie-2021-optim-elect,duan-2021-refor-mixed,tang-2019-robus-sched,ursavas-2016-optim-polic}. These problems
add additional complexities that warrant simplifications for the sake of computation. Two modes of simplification are
often found: only utilizing fast chargers during planning \cite{li-2014-trans-bus,li-2014-trans-bus,wang-2017-optim-rechar} or simplification of the charging models are made by assuming full charge
\cite{zhou-2020-bi-objec,qarebagh-2019-optim-sched,wei-2018-optim-spatio}.

Modeling the battery charge dynamics well is pertinent to this work as it directly affects the quality of the produced
schedule. Furthermore, an inaccurate model and may have detrimental affects to the health of the battery if it is
over-charged, under-charged, or forced to perform ``deep'' deep cycles \cite{zhou-2020-bi-objec,millner-2010-model-lithium,edge-2021-lithium}. While the charge profile for batteries are inherently non-linear, some
works have assumed a proportional charge increase as linear battery dynamics remain a valid assumption when the battery
SOC is below 80\% \cite{liu-2020-batter-elect}. Furthermore, other works have suggested that charging a battery nearly
to capacity is detrimental to the health and can significantly reduce the total charge cycles a battery may undergo
\cite{edge-2021-lithium,millner-2010-model-lithium}. Thus, this work assumes that a linear model is sufficiently
accurate to produce an operationally valid schedule while maintaining battery health.

Works concerning charge planning often use a version of the vehicle scheduling problem \cite{tang-2019-robus-sched,li-2014-trans-bus,he-2020-optim-charg}. Variants of this problem address infrastructure as well as determining
existing buses that should be replaced by a BEB \cite{zhou-2020-bi-objec,duan-2021-refor-mixed,rinaldi-2020-mixed-fleet,zhou-2020-collab-optim}. This work bases its implementation on what is known as the Position
allocation problem \cite{qarebagh-2019-optim-sched}. The PAP is derived from the Berth Allocation Problem (BAP) which
solves the problem of scheduling a set of vessels to be berthed and serviced. The model inputs a set of vessels arrival
and service times and outputs a schedule that defines the selected berth and the time over which it is serviced. The PAP
utilizes this model and redefines its inputs to EV arrival times and outputs queues for the EVs to be charged. While the
visits remain as discrete events, the time that the BEB is on the charger is modeled as continuous, similar to
\cite{frojan-2015-contin-berth,qarebagh-2019-optim-sched,zhou-2020-collab-optim}. Due to the close relationship
between the BAP and PAP, BAP literature may be used for the PAP. The literature shows methods of handling multiple quays
(sets of chargers) to handle general berthing scenarios \cite{frojan-2015-contin-berth,dai-2008-suppl-chain-analy}.
Heuristic procedures for quicker solve times have also been introduced \cite{imai-2001-dynam-berth}. Methods of
defining static (full time horizon) and dynamic (rolling-time horizon) models have been created for daily and real-time
solutions, respectively, and even fuzzy set theory has been applied to allow for more flexible schedules
\cite{bello-2019-fuzzy-activ,dai-2008-suppl-chain-analy,buhrkal-2011-model-discr,frojan-2015-contin-berth}. This
work utilizes an extension of the PAP as the basis of determining the feasible space of candidate solutions.

To the best of our knowledge, there is one other work that schedule BEB fleets while allowing multiple charger types,
charger, partial charging, and accounting for consumption costs \cite{whitaker-2023-a-network}. The work in
\cite{whitaker-2023-a-network} presents an optimization framework that assumed a fixed schedule, utilized non-linear
battery dynamics, partial charging, considers limited charger availability, consumption cost, and allows for multiple
charger types \cite{whitaker-2023-a-network}. This paper expands on these previous works by introducing a simulated
annealing (SA) framework that accounts for partial charging, minimizes total charger count, allows for multiple charger
types, minimizes consumption cost, and minimizes demand cost.

In what follows, the problem statement shall be provided in \ref{sec:sa-problem-description}. \ref{sec:sa-optimization-problem}
introduces the structure of the MILP formulation as well as a description of the parameters, decision variables,
objective function and constraints. In \ref{sec:sa-simulated-annealing}, the concept and theory of SA is introduced. In
particular the algorithms and methods utilized for the SA implementation for this work are discussed.
\ref{sec:sa-optimization-algorithm} outlines a generic SA algorithm, and then combines the previous sections to introduce the
particular implementation for the SA-PAP. In \ref{sec:sa-example}, an example problem is provided to demonstrate the capability
of the work provided in this paper. The results will be presented and discussed.
\section{Problem Description}
\label{sec:sa-problem-description}
Consider a fleet of BEBs scheduled to perform a set of prescribed routes on a given day. An individual BEB from said
fleet begins and completes an individual route at the same station from which it also receives its charge. During each
route, the BEB's State of Charge (SOC) is depleted by a certain amount. The charge supplied during its visit must be
enough to sustain the BEB's SOC at an appropriate level so that it may complete its next route. Provided there is a set
of chargers at the station, the bus may be placed in any single queue to receive its charge. Let the term ``arrival''
describe the time at which a BEB reaches the station. Furthermore, let the term ``visit'' denote a BEB having arrived,
awaited its predetermined time (whether it has received a charge or not), and departed from the station. Each BEB is
allowed to have multiple visits throughout the working day.

Because each bus may visit the station more than once, let the previously considered fleet contains \(n_B\) BEBs that
collectively visit a station \(n_V\) times. At said station, let there exist a pool of \(n_Q\) charging queues from which a
visiting BEB may be assigned. Upon arrival to the station, a bus is admitted to one of the \(n_Q\) queues for charging.
Each queue represents a charger that supplies the bus with a charge at a particular rate or allows the bus to sit idle
when no charging is required (i.e., a charge rate of zero). The set of possible queue indices is denoted as \(Q \in
\{1,...,n_Q\} \subset \mathbb{Z}\), where \(\mathbb{Z}\) is the set of integers. It is assumed that charger \(q \in Q\) has an associated charge rate,
denoted as \(r_q\). Let the set of arrivals be written as \(\Iset = \{ 1, ... n_V \} \subset \mathbb{Z}\), and let each BEB be prescribed
an identification number from the set \(B = \{ 1, ..., n_B \} \subset \mathbb{Z}\). As such, each visit can be represented by the tuple:
\(\visit\), in which the elements within the tuple denote the visit index, \(i \in I\), BEB identification number, \(b \in B\),
arrival time to the station, \(a \in \mathbb{R}\), departure time from the station, \(e \in \mathbb{R}\), time at which the BEB begins charging,
\(u \in \mathbb{R}\), time at which the BEB ends charging, \(d \in \mathbb{R}\), the charger queue for the BEB to be placed into, \(q \in Q\), the SOC
upon arrival, \(\eta \in \mathbb{R}\), and the index of the next visit for the currently visiting BEB, \(\xi \in I \cup \varnothing\). The null
element, \(\varnothing\), is used to specify when a BEB has no future visits. Let the set of visits be denoted as \(\I\)
where the \(i^{\text{th}}\) visit is denoted is \(\I_i\). Furthermore, let a particular item from the tuple for visit \(i\) to
be written as \(\cdot_i\). For example, the arrival time for visit \(i\) is written as \(a_i\).

The amount of time the BEB is allowed to charge during visit \(i\) is dictated by the scheduled arrival time and required
departure time, \([a_i, e_i]\). Partial charging is allowed; however, the SOC may not exceed the BEB battery capacity and
the SOC must stay above 0\%. The battery dynamics in this work is modeled as linear, which remains accurate up to about
an SOC of 80\% charge \cite{liu-2020-batter-elect}. Note that charging beyond the 80\% SOC threshold is undesirable due
to battery health concerns as higher SOC with deep cycles accelerates degradation
\cite{edge-2021-lithium,millner-2010-model-lithium}.

Each BEB arrival, except for the last arrival for each BEB, has a paired ``route'' that the BEB must perform after the
visit. This route, as one would expect, causes the BEB to discharge by some certain amount. Each bus route is assumed to
have a fixed discharge. Let the discharge of the route for visit \(i\) be denoted as \(\Delta_i \in \mathbb{R}\). Each bus has a desired
minimum battery percentage, \(\nu_b \in [0, 1]\).

\begin{table}[htbp]
\caption{\label{tab:sa-variables}Table of variables used in the paper.}
\centering
\begin{tabularx}{\textwidth}{L{0.3} L{1.2} L{0.3} L{1.2}}
\textbf{Variable} & \textbf{Description} & \textbf{Variable} & \textbf{Description}\\[0pt]
\hline
Constants &  & Constants & \\[0pt]
\(D\) & Penalty method gain factor & \(n_B\) & Number of buses in use\\[0pt]
\(\T\) & Time horizon & \(n_K\) & Number of iterations in the repetition schedule\\[0pt]
\(n_M\) & Total number of steps created by initial temperature, \(\Tau_0\), and cooling schedule & \(n_Q\) & Number of chargers\\[0pt]
\(n_V\) & Total number of visits & \(n_h\) & Number of discrete steps in time horizon\\[0pt]
\hline
Input variables &  & Input Variables & \\[0pt]
\(\Delta_i\) & Discharge of visit over after visit \(i\) & \(\alpha_b\) & Initial charge percentage time for bus \(b\)\\[0pt]
\(\epsilon_q\) & Cost of using charger \(q\) & \(\kappa_b\) & Battery capacity for each BEB\\[0pt]
\(\rho_i\) & Duration for route after visit \(i\) & \(\xi_i\) & The next index bus \(b\) will arrive\\[0pt]
\(a_i\) & Arrival time of visit \(i\) & \(\Xi_i\) & ID for bus visit \(i\)\\[0pt]
\(t_h\) & Discrete step in time horizon & \(dt\) & Discrete time slice in time horizon \(dt_h = t_h - t_{h-1}\)\\[0pt]
\(k\) & Local search iteration \(k\) & \(e_i\) & Time bus visit \(i\) must exit the station\\[0pt]
\(r_q\) & Charge rate of charger \(q\) & \(t_m\) & Element of the temperature vector created by temperature function, \(t_m \in t\)\\[0pt]
\(\nu_b\) & Minimum charge percentage allowed for each BEB &  & \\[0pt]
\hline
Direct Decision Variables &  & Direct Decision Variables & \\[0pt]
\(u_i\) & Initial charge time for visit \(i\) & \(d_i\) & Final charge time for charger for visit \(i\)\\[0pt]
\(q_i\) & Assigned queue for visit \(i\) &  & \\[0pt]
Indirect Decision Variables &  & Indirect Decision Variables & \\[0pt]
\(\eta_i\) & Charge for the bus upon arrival visit \(i\) & \(s_i\) & Amount of time spent on charger for visit \(i\)\\[0pt]
\(\sigma_{ij}\) & Binary variable determining temporal ordering of vehicles \(i\) and \(j\) & \(\psi_{ij}\) & Binary variable determining spatial ordering of vehicles \(i\) and \(j\)\\[0pt]
\(p_{d}\) & Demand cost of the schedule & \(\phi_i\) & Penalty method for visit \(i\)\\[0pt]
\(\C\) & Set of available charging times &  & \\[0pt]
\hline
\end{tabularx}
\end{table}
\section{Optimization Problem}
\label{sec:sa-optimization-problem}
The objective of this work is to present a framework that optimizes the assignment of \(n_V\) BEB visits to a set of \(n_Q\)
charging queues provided a fleet of \(n_A\) BEBs with fixed route schedules. Particularly, the framework aims to minimize
over peak power usage, energy consumption, and the total amount of chargers utilized while maintaining the SOC of each
BEB above a minimum SOC threshold.

The optimization problem outlined in this work is presented in form of an objective function with constraints. The
constraints ensure that candidate solutions are operationally feasible. The variables of optimization are to be
introduced in \ref{sec:sa-parameter-definitions} followed by a discussion of the constraints in \ref{sec:sa-constraints}. The
objective function is employed to allow relative comparisons between candidate solutions and is introduced in
\ref{sec:sa-objective-function}.

\subsection{Variable Definitions}
\label{sec:sa-parameter-definitions}
This section defines the input and decision variables used by the system. The input parameters are assumed to be fixed
prior to optimizing the system. The decision variables are the values that the SA algorithm has the freedom to
manipulate. The variables to be introduced are summarized in \ref{tab:sa-variables}.

\subsubsection{Input Parameters}
\label{sec:sa-input-variables}
The parameters are assumed to be known prior to optimization. They will be presented in two sections: battery dynamics
parameters then packing and discretization parameters. The Battery dynamic parameters are those associated with the SOC
of the BEB, the packing and discretization parameters are those that are associated with visit placement and the method
of discretizing the time horizon.

\paragraph{Battery Dynamic Parameters}
\label{sec:sa-battery-dynamic-parameters}
It is assumed that each BEB begins the working day with an initial SOC percentage of \(\alpha_b\). Let the set of initial
visits by each BEB be denoted as \(\Isetinit\) where \(\Isetinit \subset \Iset\) and the cardinality of the set is \(\lvert
\Isetinit \rvert = n_B\). The initial SOC for bus \(\Xi_i\) can be represented as \(\eta_{i} = \alpha_{\Xi_i}\kappa_{\Xi_i}; \forall i \in \Isetinit \subset
\Iset\) where \(\kappa_{\Xi_i}\) is the battery capacity for bus \(\Xi_i\). Each visit, with the exception of the final visit for each
BEB, is paired with a route that has an associated amount energy required to complete. Let the amount of energy required
to complete route \(i\) be denoted as \(\Delta_i\). As alluded to earlier, there are no routes after the last visit for each BEB.
Thus, similarly to the set of initial visits, let the set of final visits for all BEBs be denoted as \(\Isetfinal\). The
discharge for the final visit of each BEB is then defined as \(\Delta_{i} = 0; \forall i \in \Isetfinal\). After each arrival, the BEB
is assigned to a charging queue. Let \(r_q\) represent the power supplied from the charger in queue \(q \in Q\).


\paragraph{Packing and Discretization Parameters}
\label{sec:sa-packing-and-discretization-paramaters}
The cost for assigning a charger to queue \(q \in Q\) is defined by \(\epsilon_q\). \(\xi_i\) represents the next arrival index for bus
\(b_i\). As an example, suppose the ID of each BEB is recorded in order of arrival. Further suppose that \(\xi = \{ 2,1,3,2
\}\), using a starting index of 1, \(\xi_1 = 4\) as that is the next visit by bus 2. The arrival and departure times of bus
visit \(i\) to the station are denoted as \(a_i\) and \(e_i\), respectively. The notation \(t_h\) is used to denote a discrete
time that is employed to assist in the calculation of the demand cost. \(dt\) is the discrete time step \(dt = t_h -
t_{h-1}\).

\subsubsection{Decision Variables}
\label{sec:sa-decision-variables}
Decision variables are those chosen by the optimizer. The variables will be broken into two sections: direct and slack
variables. Direct decision variables are those that the system manipulates directly, and slack variables are those that
are functions of the direct.

\paragraph{Direct Decision Variables}
\label{sec:sa-direct-decision-variables}
The first two variables are \(u_i\) and \(d_i \; \forall i \in \Iset\). They represent the initial and final charging times. These
values must remain within range of the arrival and departure times, \([a_i, e_i]\), for visit \(i\). The last direct
decision variable is the queue that bus visit \(i\) can be placed in to charge, \(q_i \in \Qset\).

\paragraph{Slack Variables}
\label{sec:sa-slack-decision-variables}
Let the initial SOC for a visit be written as \(\eta_i\), where \(i \in \Iset \setminus \Iset_0\). The initial charge for visit \(i\) forms
the foundation from which the SOC of the next visit, \(\eta_{\xi_i}\), is calculated. The charge for bus \(i\)'s next visit is
equal to the initial charge for visit \(i\) plus the charge added to it by charger \(q_i\) over duration \(s_i = d_i - u_i\)
minus the discharge accumulated over route \(i\), i.e.

\begin{equation}
\label{eq:bat-chain}
  \eta_{\xi_i} = \eta_i + r_{q_i}s_i - \Delta_i\text{.}
\end{equation}

The variables \(\sigma_{ij}\) and \(\psi_{ij}\) are used to indicate whether a visit pair \((i, j)\) overlap the same space, as show
in \ref{fig:overlap}. These spatiotemporal variables uphold the following relationships:

That is, for every visit, \(\sigma_{ij} = 1 \implies\) the start charge time of visit \(j\) is greater than the end charge time
of visit \(i\). Similarly, \(\psi_{ij} = 1 \implies\) the queue for visit \(j\) is of a greater index than visit \(i\). A value of
zero for either of these variables conveys no information.

The variable \(\C\) is the set that describes the availability for all chargers. That is, \(\C\) is a set of \(n_Q\) sets that
contain available charger times for each queue \(q \in Q\). Let a set of available charge times for queue \(q\) be defined as
\(\C_q\).

\subsection{Objective Function}
\label{sec:sa-objective-function}
This work aims to minimize the total ``cost'' of utilizing a given charge schedule. Let \(J(\I)\) represent the objective
function. The objective function for this problem has four main considerations: charger assignment, consumption cost,
demand cost, and penalty for an insufficient initial SOC. Each of which will be discussed in turn in the subsequent
sections.

\subsubsection{Assignment Cost}
\label{sec:sa-assignment-cost}
The assignment cost represents the costs of assigning a bus to a particular queue. This is done as a method of
minimizing the total utilized chargers. The assignment cost is written as

\begin{equation}
\label{eq:assignment-cost}
\sum_{i=1}^{n_V} \epsilon_{q_i}r_{q_i}\text{.}
\end{equation}

This cost is effectively the cost on the choice of \(q\). Recalling the form of \(Q\), particularly the ordering in which
the set was defined. Taking \(\epsilon\) to be constructed using the same ordering (idle, slow, then fast charging queues), let
the first \(n_B\) queues have no cost. Furthermore, let the next \(n_Q\) charging queues be of the form \([P, 2P, ...,
n_QP]\). Concatenating these vectors yields \(\epsilon = [[0; n_B], [P, 2P, ..., Pn_Q]]\), where \([0; n_B]\) is used to denote a
vector populated with zeros of length \(n_B\). In words, this form accrues no cost when assigning a BEB to an idle queue
while still minimizing charger count and encouraging the use of slow chargers over fast. Thus, the larger the index of
\(q\), the larger the cost. The \(\epsilon\) vector described above is one of many forms that the vector may take; however, form
shown described is what is applied in this work.

\subsubsection{Penalty Method}
\label{sec:sa-penalty-method}
A penalty method is to be implemented in the objective function that is enabled when the \(\eta_i\) falls below a defined
threshold. Let the piecewise function that enables/disables the penalty method be of the form

\begin{equation}
\label{eq:penalty}
  \phi(x) =
  \begin{cases}
    0   & x \ge 0 \\
    x^2 & x < 0\\
  \end{cases}
\end{equation}

Furthermore, letting \(x = \eta_i - \nu_{\Xi_i} \kappa_{\Xi_i}\), where \(\nu_{\Xi_i} \kappa_{\Xi_i}\) is the minimum charge threshold, applies a penalty
proportional to the difference of the SOC and the threshold squared.

\begin{figure}[htpb]
  \centering \includegraphics{img/overlap}
  \caption{Examples of different methods of overlapping. Space overlap: $q_{k_1} > q_{i} + 1 \therefore \psi_{ik_{1}} = 1$.
    Time overlap $u_{k_2} < u_{j} + s_j \therefore \sigma_{k_{2}j} = 0$. Similarly, $\sigma_{k_3 i} = 0$.}
  \label{fig:overlap}
\end{figure}

Using the form of \ref{eq:penalty} with and added scalar, \(D\), is employed so that the cost of deviating from the threshold
heavily influences the outcome of the objective function. Therefore, the penalty method is written as

\begin{equation}
\label{eq:penalty-method}
\sum_{i=1}^{n_V} D \phi_i(\eta_i - \nu_{\Xi_i} \kappa_{\Xi_i})\text{.}
\end{equation}

\subsubsection{Consumption Cost}
\label{sec:sa-consumpction-cost}
In most cases, energy companies rely on a volumetric rate as a method of track customer electricity consumption (i.e.
total electricity consumed over a billing period). As such, a method of reducing the total energy consumed by the system
is desired. The energy total energy consumed by a charge schedule is defied by what is known as the consumption cost.
The consumption cost is the summation of all the energy being used over all the active periods for each charger in the
time horizon. This is represented by the summation

\begin{equation}
\label{eq:consumption-cost}
\sum_{i=1}^{n_V} r_{q_i}s_i\text{.}
\end{equation}

That is, the charge rate, \(r_{q_i}\), for the active charger, \(q_i\), is multiplied by the time that the charger will be
utilized, \(s_i\).

\subsubsection{Demand Cost}
\label{sec:sa-demand-cost}
Historically for large industrial customers, energy companies often further rely on a demand cost in conjunction with a
consumption cost for billing. The consumption cost is an important metric as it measures how much power a customer may
require over billing period. Energy companies, having to potentially meet large peaks in demand, offset that cost to the
customer. Thus, when a demand and consumption costs are imposed, not only is the total energy reduction desirable, but
also how much power is consumed at a given moment.

A method of calculating the demand charge is done by calculating the average power consumption over a given period of
time. Let the average power used over an arbitrary interval, \(T_p\), be represented by

\begin{equation}
\label{eq:p}
p_{T_p}(t) = \frac{1}{T_p} \int_{t-T_p}^{t} p(\tau) d\tau\text{.}
\end{equation}

Energy companies take the largest peak when calculating the demand cost. Therefore, let the cost of the peak power
consumption be dictated by the maximum average power:

\begin{equation}
\label{eq:pmax}
p_{max}(t) = \max\limits_{\tau \in [0,t]}p_{T_p}(\tau)\text{.}
\end{equation}

Furthermore, a fixed minimum average power is introduced that is intended to act as a base threshold before the cost
begins to increase. Let this fixed threshold be defined as \(p_{fix}\), the demand cost is calculated using

\begin{equation}
\label{eq:pdem}
p_d(t) = \max(p_{fix},p_{max}(t))\text{.}
\end{equation}

Hence, \ref{eq:pdem} defines a cost beginning with a value of \(p_{fix}\) from which it may only increase if \(p_{15}(t) >
p_{fix}\).

Although the charge times for each BEB are continuous, due to the discrete nature of visits it is simpler to determine a
vector of discrete power consumption over the time horizon from which the average power demand cost may be derived. To
discritize \(p_d\), let \(h \in \{ 1, 2, ..., n_H \} \subset \mathcal{Z}\) where \(n_H\) is the total number of steps. Furthermore, let \(p\)
define the vector of discrete power demand over the time horizon and let \(p_h \in p\) be the demand over time step \(h\). For
conciseness of notation \(t_h\) will be abused to denote the time in discrete form (as opposed to \(t\) being continuous)
and let \(dt = t_h - t_{h-1}\). Each entry \(p_h \in p\) is found by taking the summation of \(r_{v_i}\) for all the chargers
active during the time interval \([t_h, t_h + dt]\). Thus, having an estimate of the demand over each step, the average
demand can be calculated by taking an average of the Riemann sum over each \(T_p\) interval as follows:

\begin{equation}
p_{T_p}[h] = \frac{1}{T_p} \sum_{h-\frac{T_p}{dt}+1}^h p_h,
\end{equation}

where \(T_p \le h \le n_H\). Similarly to before, the maximum \(p_{T_p}[h]\) value is to be retained via \(p_{max} =
\max\limits_{h \in H}p_{T_p}[h]\). Thus, the discrete demand cost is expressed as

\begin{equation}
\label{eq:pd-dis}
  p_d = \max(p_{fix}, p_{max})\text{.}
\end{equation}

To conclude this section, the objective function is to written in its entirety:

\begin{equation}
\label{eq:objective-function}
  J(\I) = p_d + \sum_{i=1}^{n_V} \epsilon_{q_i}r_{q_i} + D \phi_i(\eta_i - \nu_{\Xi_i} \kappa_{\Xi_i}) + r_{q_i}s_i\text{.}
\end{equation}

\subsection{Constraints}
\label{sec:sa-constraints}
While the objectives are used to compare solutions, constraints are introduced to ensure that the solutions are
operationally valid. Operationally validity requires that allocated BEBs do not overlap spatially or temporally.
Furthermore, the SOC of a bus at a particular visit is related to the charge its previous visit by the amount of
charging and discharging that has occurred. Finally, buses must leave the charger before their scheduled departure time.
These constraints are represented as follows:

\begin{multicols}{2}
\begin{subequations}
\label{eq:constraints}

  \begin{equation}
      \label{seq:c0}
      u_j - d_i - (\sigma_{ij} - 1)T \ge 0
  \end{equation}
  \begin{equation}
      \label{seq:c1}
      q_j - q_i - 1 - (\psi_{ij} - 1)Q \ge 0
  \end{equation}
  \begin{equation}
      \label{seq:c2}
      \sigma_{ij} + \sigma_{ji} \le 1
  \end{equation}
  \begin{equation}
     \label{seq:c3}
      \psi_{ij} + \psi_{ji} \le 1
  \end{equation}
  \begin{equation}
      \label{seq:c4}
      \sigma_{ij} + \sigma_{ji} + \psi_{ij} + \psi_{ji} \ge 1
  \end{equation}
  \begin{equation}
      \label{seq:c5}
      s_i = d_i - u_i
  \end{equation}
  \begin{equation}
      \label{seq:c6}
       \eta_{\xi_i} = \eta_{i} + r_{q_i}s_i - \Delta_i
  \end{equation}
  \begin{equation}
      \label{seq:c7}
      \kappa_{\Xi_i} \geq \eta_{i} + r_{q_i}s_i
  \end{equation}
  \begin{equation}
      \label{seq:c8}
      a_i \leq u_i \leq d_i \le e_i \le \T
  \end{equation}
\end{subequations}
\end{multicols}

Constraints \ref{seq:c0}-\ref{seq:c4} are the ``queuing constraints''. They prevent overlap both spatially and temporally
as shown in \ref{fig:overlap}. The y-axis represents the possible queues for a bus visit to be placed into, and the
x-axis represents the time that can be reserved for each visit. The shaded rectangles represent time that has been
scheduled in the horizontal direction, and the queue allocated for each bus visit in the vertical direction. In other
words, the set of constraints \ref{seq:c0} - \ref{seq:c4} aim to ensure that these shaded rectangles never overlap.

Constraint \ref{seq:c0} states that the starting charge time for BEB \(u_j\) must begin after the previous BEB departs,
\(d_i\). The constraint utilizes big-M notation to activate or deactivate the constraint. A value of \(\sigma_{ij} = 1 \implies\)
bus \(i\) has detached from the charger before bus \(j\) has begun charging. If \(\sigma_{ij} = 0\), then the constraint is of the
form \(\T + d_i > u_j\) rendering the constraint ``inactive'' because \(u_j\) cannot be larger than \(\T + d_i\). Similarly, for
\ref{seq:c1}, \(\psi_{ij}\) determines whether the vehicles are charging in the same queue. A value of \(\psi_{ij} = 1 \implies\)
BEB \(i\) is in a queue index that is less than BEB \(j\). If \(\psi_{ij} = 0\) then the constraint is deactivated. Constraints
\ref{seq:c2} - \ref{seq:c4} enforce spatial and temporal ordering between each queue/vehicle pair. \ref{seq:c2} and
\ref{seq:c3} ensure that BEB \(i\) is not placed before and after \(j\) spatially, temporally, or both because that is
impossible. \ref{seq:c4} ensures that BEB \(i\) and \(j\) do not have scheduling conflicts spatially or temporally.

 \ref{seq:c5} describes the service time of the bus. \ref{seq:c6} calculates the initial charge for the next visit for
bus \(b_i\). \ref{seq:c7} ensures that the bus is not being over-charged. \ref{seq:c8} ensures the continuity of the times
(i.e. the arrival time is less than the initial charge which is less than the detach time which is less than the time
the bus exits the station and all must be less than the time horizon).
\section{Simulated Annealing}
\label{sec:sa-simulated-annealing}
SA is a well-studied local search metaheuristic used to solve various optimization problems
\cite{gendreau-2018-handb-metah,press-1992-numer-recip}. A metaheuristic is a high-level problem-independent algorithm
framework that provides a set of guidelines or strategies to develop heuristic optimization algorithms
\cite{radosavljevic-2018-metah-optim}. That is, metaheuristic strategies provides guidelines for implementation;
however, each problem must tailor its implementation to meet its particular needs.

SA is an exploitation oriented, single-solution based metaheuristic. In addition to the advantages of simplicity, both
theoretically and in its implementation \cite{gendreau-2018-handb-metah,radosavljevic-2018-metah-optim}. SA is also of
considerable interest for global optimization over regions containing several local and global minima due to inherent
non-linearities of the objective function \cite{gendreau-2018-handb-metah}. Note that because SA is a metaheuristic
strategy, there is no guarantee of optimality, its objective is to yield a result that is ``good enough''
\cite{radosavljevic-2018-metah-optim}. This model is named after its analogized process where a crystalline solid is
heated then allowed to cool at a slow rate until it achieves its most regular possible crystal lattice configuration
(i.e. its minimum lattice energy state) \cite{henderson-1989-theor-pract,press-1992-numer-recip}. SA establishes a
connection between this thermodynamic process and the search for global optima in optimization problems.

There are three key components to SA: cooling schedule (temperature function), acceptance criteria, and generation
mechanisms \cite{keller-2019-multi-objec,press-1992-numer-recip}. The temperature function describes the speed at
which the system is ``cooled'' over each iteration. The term ``system'' refers to a general instance of SA. The generation
mechanisms provide a means of modifying the system by some singular discrete change that is within the neighborhood of
the previous solution \cite{gendreau-2018-handb-metah}. The acceptance criteria is a function of the system temperature
that makes the decision whether the system will accept an inferior solution in favor of exploring the solution space.
Finally, the constant temperature iteration count is the number of steps taken to try to exploit a solution at a
constant temperature. Each of these mechanisms are elaborated in the subsequent sections.

\subsection{Cooling Equation}
\label{cooling-equation-experimental}
The temperature function models a ``rate of cooling'' for the SA process. Initially, when the temperature is high, SA
encourages exploration. As the process begins to ``cools down'' (in accordance to the cooling schedule), it begins to
encourage local exploitation of the solution (rather than exploration)
\cite{rutenbar-1989-simul-anneal-algor,henderson-1989-theor-pract}. There are three common basic types of cooling
equations: linear, geometric, and exponential. Geometric cooling schedules are most widely used in practice
\cite{keller-2019-multi-objec}. As such, it will also be employed by this work. It is defined by the difference
equation

\begin{equation}
\label{eq:cool}
t_m = \beta t_{m-1}\text{.}
\end{equation}

The value of \(\beta\) may vary anywhere between the range \([0,1)\). The further \(\beta\) is from 1, the quicker the function
converges to zero. \ref{fig:geometric} demonstrates this principle by plotting the geometric schedule using varying
values of \(\beta\).

\begin{figure}[t!]
  \centering \includegraphics[width=0.6\textwidth]{img/geometric.png}
  \caption{Geometric cooling schedule utilizing various value of $\beta$.}
  \label{fig:geometric}
\end{figure}

\subsection{Acceptance Criteria}
\label{sec:sa-acceptance}
In SA, the algorithm stores a candidate solution that is continuously compared to newly generated solutions. Let the
stored solution be referred to as the ``active solution''. During each iteration, a new candidate solution is generated
and compared to the active solution to determine if the new solution should replace the active solution. To determine if
the active solution is to be replaced, an acceptance criterion is defined. In an effort to encourage exploration,
inferior candidate solutions have a probability of being accepted. The probability of accepting an inferior candidate
solution is described by the function \(\exp(-\frac{J(\I) - J(\bar{\I})}{t_m})\) where \(J(\cdot)\) is the objective function
(\ref{sec:sa-objective-function}), \(t_m\) is current temperature, \(\I\) is the current solution, and \(\bar{\I}\) is the new
candidate solution. Formally, let \(\Delta E \equiv J(\I) - J(\bar{\I})\) and let \(f(\cdot)\) be the function that describes the
probability of accepting a candidate solution \(\bar{\I}\). The probability of accepting a candidate solution is thus of
the form \cite{keller-2019-multi-objec}

\begin{equation}
\label{eq:candaccept}
f(\I,\bar{\I},t_m) =
\begin{cases}
  1                   & \Delta E > 0 \\
  e^{- \frac{\Delta E}{t_m}} & \text{otherwise}
\end{cases}\text{.}
\end{equation}

\subsection{Neighbor Generators and Wrappers}
\label{sec:sa-generation-mechanisms}
Generation mechanisms are used to create a neighboring candidate solution \cite{gendreau-2018-handb-metah}. That is,
the generating function creates a solution that can be reached in a single iteration from the active solution. In the
case of the problem statement made in \ref{sec:sa-problem-description}, five primitive generation mechanism are used: new visit,
slide visit, new charger, new window, wait. The purpose of each of these generators is to assign new visits to a
charger, adjust a bus visits initial and final charge time within the same time frame/queue, move a BEB from one charger
to another with the same charge schedule, move a bus to its idle queue. Each generator will be discussed in more detail
in \ref{sec:sa-generators}.

These generation mechanisms will in turn be utilized by two wrapper functions. The schedule generation is to used create
an initial candidate solutions for SA and the perturb schedule generator is used to take a candidate solution and alter
it slightly in an attempt to step toward a global or local minimum. The wrapper functions will be discussed in
\ref{sec:sa-generator-wrappers}. However, prior to discussing the primitives and wrapper generating functions, their respective
inputs and outputs must be defined.

\subsubsection{Generator Input/Output}
\label{sec:sa-generator-input-output}
This section discusses inputs and outputs of each generator. The input consists of the bus visit index of interest, the
current state of visits, \(\I\), and the current state of the charger availability, \(\C\). The output of each generator
affects a subset of \(\I\) and the updated charger availability \(\C\).

\paragraph{Generator Input}
\label{sec:orgf5c5d80}
Each generator accepts a tuple \(\Sol \equiv (i, \I, \C)\) where \(i\) is the visit index being manipulated, \(\I\) is the set of
visits, and \(\C\) is the set that describes the availability for all chargers \(q \in \Qset\).

\paragraph{Generator Output}
\label{sec:org7cc76e5}
The output of the generating functions is the same as the input, but with changes applied to it by a generator. Let a
modified variable be denoted with a bar, \(\bar{\cdot}\). Thus, the modified input tuple is written as \(\bar{\Sol}\). Although
not all the variables in \(\Sol\) are modified, it is written in this manner for the sake of consistency and simplicity in
bookkeeping.

\subsubsection{Generators}
\label{sec:sa-generators}
This section describes and outlines the different generator types. Recall that to satisfy constraints, \(n_B\) extra idle
queues are added that provide no power to the BEB. Because of this, the set of queues is fully defined where \(Q\) is the
ordered set of idle queues, slow queues, then fast queue. The use case for the idle queues are for when a bus is not to
be placed on a charger. Rather, it will be placed in the queue, \(q \in B\), which satisfies the previously defined spatial
constraints while allowing the bus to be ``set aside''. The charge queues are denoted by \(q \in \{1, ..., n_B , n_B + 1,
..., n_Q\}\).

In the development of the algorithms, the dot notation is to be introduced to extract variables from tuples. For
example, suppose the arrival time is desired to be extracted from visit \(i\). Given \(\I\), the notation that describes
extracting the initial charge time for visit \(i\) is written as \(u_i \equiv \I_{i.u}\).

\paragraph{New visit}
\label{sec:sa-new-visit}
The new visit generator defined in \ref{alg:new-visit} describes the process of moving a BEB, \(b \in B\), from a waiting
queue, \(q \in B\), to a charging queue, \(q_i \in \{n_B + 1, ..., n_Q\}\), within its arrival/departure time \([a, e]\). Let
\(\U_{\{\cdot\}}\) indicate that an element is selected randomly with a uniform distribution from the set \(\{\cdot\}\). For
example, \(\U_{[a, e]}\) indicates that a value will be selected between \(a\) and \(e\) with a uniform distribution.
\ref{alg:new-visit} begins by extracting variables. Lines 6 and 7 randomly select a charging queue and available time
frame with a uniform distribution, respectively. Line 8 attempts to assign the visit the time frame found in Line 7, if
it succeeds, the updated visit is returned. Otherwise, the null value is returned.

The function \texttt{findFreeTime} is the algorithm that determines whether a visit's time at the station \([a, e]\) can be placed
in the time availability of charger \(q\). Let the available time for charger \(q\) for visit \(i\) be denoted as \(C \equiv
\C_{i.q}\). Furthermore, let the lower and upper bound of \(\C\) be denoted as \(\C_L\) and \(C_U\), respectively. The
algorithm checks whether the BEB time at the station, \([a_i, e_i]\) fits within the charger availability \([C_L, C_U]\). If
it does, a random charge time slice is returned, otherwise the null value is returned.

\begin{algorithm}[H]
  \scriptsize
  \caption{New visit algorithm}
  \label{alg:new-visit}
  \LinesNumbered
  \TitleOfAlgo{New Visit}
  \KwIn{$\Sol$}
  \KwOut{$\bar{\Sol}$}

  \SetKwFunction{Union}{Union}
  \SetKwFunction{findFreeTime}{findFreeTime}

  \Begin
    {
      $i \leftarrow \Sol_{i}$\tcc*{Extract visit index}
      $a \leftarrow \I_{i.a}$\tcc*{Extract the arrivial time for visit $i$}
      $e \leftarrow \I_{i.e}$\tcc*{Extract the departure time for visit $i$}
      $q \leftarrow \I_{i.q}$\tcc*{Extract the current charge queue for visit $i$}
      $\bar{q} \leftarrow \mathcal{U}_{Q}$\tcc*{Select a random charging queue with a uniform distribution}
      $C \leftarrow \mathcal{U}_{\C_q}$\tcc*{Select a random time slice from $\C_q$}

      \If(\tcc*[f]{If there is time available in $C_q^j$}){($\bar{C}, \bar{u}, \bar{d}$) $\leftarrow$ \findFreeTime{$C, i, q, a, e$} $\not\in \varnothing$}
         {
           \Return{($i, (\bar{q},\bar{u},\bar{d}),\bar{C}$)}\tcc*[f]{Return visit}
         }

         \Return{($\varnothing$)}\tcc*{Return nothing}
    }
\end{algorithm}

\paragraph{Slide visit}
\label{slide-visit}
This primitive generator is used for visits that have already been scheduled. Because of the constraint \ref{seq:c8}
there may be some slack to manipulate \([u_i, d_i]\) within the window \([a_i, e_i]\). That is, two new values, \(u_i\) and
\(d_i\) are randomly selected with a uniform distribution that satisfy the constraint \(a_i \leq u_i \leq d_i \leq e_i\). Line 2 of
\ref{alg:slide-visit} purges the visit from the charger availability schedule. The \texttt{Purge} function simply removes an
assigned charge time from the set \(\C\). Without altering selected queue, the charge time randomly re-assigned with a
uniform distribution. Upon success, the updated tuple is returned, otherwise the null value is returned.

\begin{algorithm}[H]
  \scriptsize
  \caption{Slide Visit Algorithm} \label{alg:slide-visit}
  \LinesNumbered
  \TitleOfAlgo{Slide Visit}
  \KwIn{$\Sol$}
  \KwOut{$\bar{\Sol}$}

    \SetKwFunction{Purge}{Purge}

    \Begin
    {
      $(i, \I, \bar{\C}) \leftarrow$\Purge{$\Sol$}\tcc*{Purge visit $i$ from charger availibility matrix}
      $C \leftarrow \bar{C}_{i.q_i}$\tcc*{Get the time availability of the purged visit}

      \tcc{If there is time available in $C$}
      \If{($\bar{C}, \bar{u}, \bar{d}$) $\leftarrow$ \findFreeTime{$C$, $\Sol_i$, $\I_q$, $\I_{i.a}, \I_{i.e}$} $\not\in \varnothing$}
      {
        \Return{($i, \I, (\I_{i.q_i},\bar{u},\bar{d}),\bar{C}$)}\tcc*[f]{Return updated visit}
      }

        \Return{($\varnothing$)}\tcc*{Return nothing}
    }
  \end{algorithm}

\paragraph{New charger}
\label{new-charger}
The new charger generator moves a visit \(\I_i\) to a new charging queue while maintaining the same charge time, \([u_i,
d_i]\). \ref{alg:new-charger} initial purges the visit from the charger availability set, a queue is selected at random
with a uniform distribution, then the new selection is checked whether the charge time \([u_i, d_i]\) may be assigned to
the new queue.

\begin{algorithm}[H]
  \scriptsize
  \caption{New Charger Algorithm} \label{alg:new-charger} \LinesNumbered \TitleOfAlgo{New Charger} \KwIn{$\Sol$}
  \KwOut{$\bar{\Sol}$}

    \SetKwFunction{Purge}{Purge}

    \Begin
    {
      $(i, \I, \bar{\C}) \leftarrow$\Purge{$\Sol$}\tcc*{Purge visit $i$ from charger availibility matrix}
      $q \leftarrow \mathcal{U}_{Q}$\tcc*{Select a random charging queue with a uniform distribution}

      \If(\tcc*[f]{If there is time available in $C_{q}$}){($\bar{C}, \bar{u}, \bar{d}$) $\leftarrow$ \findFreeTime{$\bar{\C}_{i.q}$, $\Sol_i$, $\I_q$, $\I_{i.a}, \I_{i.e}$} $\not\in \varnothing$}
      {
        \tcc{Return visit, note $u$ and $d$ are the original inital/final charge times.}
        \Return{($i, \I, (q,\I_{i.u}, \I_{i.d}),\bar{\C}$)}
      }

      \Return{($\varnothing$)}\tcc*{Return nothing}
    }
  \end{algorithm}

\paragraph{Wait}
\label{sec:sa-wait}
The wait generator simply removes a bus from a charger queue and places it in its idle queue, \(q_i \in B\). \ref{alg:wait}
begins by purging the visit from the charger availability set, the visit is then assigned to its idle queue for the
duration of its time at the station.

\begin{algorithm}[H]
\scriptsize
\caption{Wait algorithm} \label{alg:wait}
    \LinesNumbered
    \TitleOfAlgo{Wait}
    \KwIn{$\Sol$}
    \KwOut{$\bar{\Sol}$}

    \SetKwFunction{Purge}{Purge}

    \Begin
    {
      $(i, \I, \bar{\C}) \leftarrow$\Purge{$\Sol$}\tcc*{Purge visit $i$ from charger availibility matrix}
      $\bar{\C}'_{\I_{i.\Gamma_i}} \leftarrow \C' \cup \{[\I_{i.a}, \I_{i.e}]\}$\tcc*{Update the charger availability matrix for wait queue $\bar{\C}_{i.q_i}$}
      \Return{$(i, \I, (\I_{i.b}, \I_{i.a}, \I_{i.e}), \bar{\C})$}\tcc*[f]{Return visit}
    }
  \end{algorithm}

\paragraph{New Window}
\label{sec:sa-new-window}
New window, as shown in \ref{alg:new-window}, is a combination of \ref{alg:new-visit} (new visit) and \ref{alg:wait}
(wait). By this it is meant that visit \(i\) is placed in its wait queue then added back in as if it were a new visit.
This implies that the BEB may be assigned to a different queue and a new charge time slice. \ref{alg:new-window} begins
by purging the visit from the charger availability set. \ref{alg:wait} is executed, upon success, \ref{alg:new-visit} is
executed. If that succeeds, return the updated tuple, otherwise return the null value.

\begin{algorithm}[H]
  \scriptsize
  \caption{New window algorithm} \label{alg:new-window}
  \LinesNumbered
  \TitleOfAlgo{New Window}
  \KwIn{$\Sol$}
  \KwOut{$\bar{\Sol}$}

  \SetKwFunction{NewVisit}{NewVisit}
  \SetKwFunction{Wait}{Wait}

  \Begin
  {
    $\bar{\Sol} \leftarrow$\Wait{$\Sol$}\tcc*{Assign visit to its respective idle queue}
    \If(\tcc*[f]{Add visit $i$ back in randomly})
       {
         $\bar{\bar{\Sol}} \leftarrow$ \NewVisit{$\bar{\Sol}$} $\not\in \varnothing$
       }
       {
         \Return{$\bar{\bar{\Sol}}$} \tcc*[f]{Return visit}
       }

       \Return{($\varnothing$)}\tcc*{Return nothing}
  }
\end{algorithm}

\subsubsection{Generator Wrappers}
\label{sec:sa-generator-wrappers}
This section covers the algorithms utilized to select and execute different generation processes. The generator wrappers
are the methods immediately called by the SA algorithm. Each wrapper utilizes the primitive generators previously
described and returns either a new charge schedule or a modified charge schedule.

\paragraph{Charge Schedule Generation}
\label{sec:sa-charge-schedule-generation}
The objective of \ref{alg:charge-schedule-generation} is to assign each visit to a random charge queue and charge time.
Specifically, this generator exists to initialize the system with a solution in a greedy manner.
\ref{alg:charge-schedule-generation} loops through each visit and executes \ref{alg:new-visit} to place visit \(i\) at
random queue with a random charge time.

\begin{algorithm}[H]
\scriptsize
\caption{Charge schedule generation algorithm} \label{alg:charge-schedule-generation}
    \LinesNumbered
    \TitleOfAlgo{Candidate Solution Generator}
    \KwIn{$\Sol$}
    \KwOut{$\bar{\Sol}$}

    \SetKwFunction{NewVisit}{NewVisit}

    \Begin
    {
        \tcc{Select an unscheduled BEB visit from a randomly indexed set of visits}
        \ForEach {$\I_i \in \I$}
        {
            ($i, \bar{\I}$, $\bar{\C}$) $\leftarrow$ \NewVisit{($\I_i$, $\I$, $\C$)}\tcc*{Assign the bus to a charger}
        }
            \Return{($0, \bar{\I}$, $\bar{\C}$)}
    }
  \end{algorithm}

\paragraph{Perturb Schedule}
\label{sec:sa-tweak-schedule}
Once the active solution has been created by \ref{alg:charge-schedule-generation}, the SA process begins modifying it to
create candidate solutions. After each step of the cooling function, the active solution will be altered \(n_K\) times by
a random primitive generator. During these \(n_K\) iterations the active solution is modified to create a neighboring
candidate solution. This candidate solution will then be compared against the active solution in the manner discussed in
\ref{sec:sa-acceptance}. \ref{alg:perturb-schedule} describes the method by which the SA algorithm decides how to perturb the
schedule. The method that will be employed generate a neighboring solution is as follows: pick a visit, pick a primitive
generator, and execute said primitive generator once. Let \(\W^y_{[\cdot]}\) denote a random selection with a distribution
specified by a weight vector \(y \in \mathbb{R}\). Thus, \ref{alg:perturb-schedule} is as follows: select a visit with a uniform
distribution, select a primitive with a weighted distribution. Letting \(n_G\) denote the number of primitive generating
functions, the selected primitive with a weighted distribution is denoted as \(\W^y_{[1, n_G]}\). The primitive is then
executed, and the results are returned.

\begin{algorithm}[H]
\scriptsize
\caption{Perturb schedule algorithm} \label{alg:perturb-schedule}

    \LinesNumbered
    \TitleOfAlgo{Perturb Schedule}
    \KwIn{$\Sol$}
    \KwOut{$\bar{\Sol}$}

    \SetKwFunction{PGF}{PGF}

    \Begin
    {
        $\I_i\leftarrow\; \U_{\I}$\tcc*{Randomly select a visit}
        $i \leftarrow\; \I_i$\tcc*{Extract visit index}
        $y \leftarrow [y_1, y_2, ...]$\tcc*{Define the weight of each primitive generator}
        $PGF \leftarrow\; \W^y_{[1,n_G]}$\tcc*{Select one of the generator functions}
        $\bar{\Sol} \leftarrow$ \PGF{($i$, $\I$, $\C$)}\tcc*{Excecute the generator function}
        \Return{($0, \bar{\I}$, $\bar{\C}$)}
    }
\end{algorithm}

\subsection{Alternative Heuristic Implementation}
\label{sec:sa-heuristic-implementation}
As suggested by the works in \cite{Zhang_2010,Xinchao_2011}, applying heuristics to the generating functions can
manipulate the searched neighborhoods in a way that may assist the SA algorithm with convergence. As a test to assist in
minimizing charger utilization, a simple heuristic was applied to \ref{alg:new-visit} and \ref{alg:new-charger} in the
method that they select new charging queues. Suppose rather than selecting a queue at random from \(q \in Q\), the
algorithms randomly select whether to place a BEB in a slow or fast charging queue with a weighted distribution favoring
slow chargers. Once the charger type has been selected, the algorithm will then begin incrementally attempting to place
the BEB in a queue of that type beginning from the smallest index of that charger type. For example, if a BEB has been
selected to be placed in a queue with a slow charger, the algorithm begins by attempting to place the BEB in the charger
queue \(q = n_B + 1\). If it is unable to be placed in that queue, it then attempts to be placed in the next queue \(q =
n_B + 2\). This is done incrementally until all the queues have been exhausted. At the expense of an additional up-front
computation cost, the heuristic will attempt to pack the visit optimally in the spacial sense.
\section{Optimization Algorithm}
\label{sec:sa-optimization-algorithm}
This section combines the generation algorithms and the optimization problem into a single algorithm (\ref{alg:sa-pap}).
While the SA PAP generally is written almost identically to that of the general SA algorithm, the general SA assumes
that the generated candidate solutions are in the solution space of the problem, \(\omega \in S\) where \(S\) is the solution
space. Initialization and the perturbation of a schedule must be verified to ensure that the generated schedule is in
the solution space. Therefore, the objective function and constraints introduced in \ref{sec:sa-constraints} and
\ref{sec:sa-objective-function}, respectively, must be employed to verify that the output of
\ref{alg:charge-schedule-generation} is in the feasible space, \(S\).

As previously stated, the generating functions directly influence the values of the assigned charge queue, charge
initialization time, and charge completion time: \(q_i\), \(u_i\), and \(d_i\), respectively. Having generated those values,
the rest of the decision variables may be derived. Let's begin by reviewing over the packing constraints.
\ref{seq:c0}-\ref{seq:c1} are employed to enable and disable \(\sigma_{ij}\) and \(\psi_{ij}\) and \ref{seq:c2}-\ref{seq:c4} ensure
the validity of the values. \ref{seq:c5} can be directly calculated and \ref{seq:c8} is fully defined.

Changing the focus over to the dynamic constraints, similarly to what was seen with the packing constraints, the battery
dynamic constraints are also fully defined and can be calculated. \ref{seq:c6} is sequentially calculated after a given
schedule has been fully defined. \ref{seq:c7} is evaluated to ensure the BEB is not overcharged. The penalty method
implemented in \ref{sec:sa-objective-function} is set in place to allow the SOC to go below the specified threshold, \(\nu_{\Xi_i}
\kappa_{\Xi_i}\), but punish the solution for doing so. Thus, over time, the candidate solutions will be encouraged toward a
solution that does not activate the penalty method (i.e. is solution is truly feasible).

The SA-PAP algorithm in \ref{alg:sa-pap} will now be outlined. The algorithm begins be creating a temperature schedule
and creating an initial solution. The algorithm then begins to iterate through the temperature schedule (outer loop).
For each iteration of the outer loop, an inner loop is executed \(n_K\) times. During this inner loop, the solution is
modified by a generating function to create a candidate solution. The candidate is solution is then compared with the
active solution, and updated according to the acceptance criteria. These actions are performed until the temperature
function is exhausted.

\begin{algorithm}[H]
  \scriptsize
  \caption{Simulated annealing approach to the position allocation problem} \label{alg:sa-pap}
  \LinesNumbered
  \TitleOfAlgo{SA PAP}
  \KwIn{($\I$ , $\C$)}
  \KwOut{($\bar{\I}$, $\bar{\C}$)}

  \SetKwFunction{Temp}{$\Tau$}
  \SetKwFunction{CSG}{CSG}
  \SetKwFunction{PS}{PS}
  \SetKwFunction{Obj}{J}

  \Begin
    {
      \tcc{Generate vector of temperatures given temperature function $\Tau$ and initial temperature $\Tau_0$}
      $t \leftarrow$ \Temp{$\Tau_0$}

      $\Sol \leftarrow$\CSG{($\I$, $\C$)}\tcc{Generate an initial solution}

      \tcc{For each item in the temperature vector}
      \ForEach{$t_k \in t$}
       {
        \tcc{For each step in the constant temperature repitition counter}
        \ForEach{$k \in \{0, 1, ..., n_K\}$}
        {
          $\bar{\Sol} \leftarrow$ \PS{($\I$, $\C$)} \tcc*{Generate a new solution}
          $\Delta E = $ \Obj{$\bar{\Sol}_{\I}$}  - \Obj{$\Sol_{\I}$} \tcc*{Calculate the difference of fitness scores}

          \If{$\bar{\I} \in S$ and $\Delta E < 0$}{$\Sol \leftarrow \bar{\Sol}$}
          \If{$\bar{\I} \in S$ and $\Delta E \ge 0$}{$\Sol \leftarrow \bar{\Sol}$ with probability $e^{\frac{\Delta E}{t_k}}$}
        } % For k
      }   % For t_k \in t

      \Return{($\I$ , $\bar{\C}$)}
    } % Begin
\end{algorithm}

\section{Example}
\label{sec:sa-example}
An example is now provided to demonstrate the utility of the developed SA charge scheduling technique. In
\ref{sec:sa-beb-scenario} a description of the example scenario is presented followed by a brief introduction of the original
MILP PAP. An alternative heuristic based planning strategy called Qin-Modified, and a heuristic modification to the SA
PAP are also used as comparisons to the SA PAP technique presented in this work. \ref{sec:sa-results} presents the results for
each of planning strategies. The results are also analyzed and discussed.

\subsection{BEB Scenario}
\label{sec:sa-beb-scenario}
The test scenario was run over a time horizon of \(T=24\) hours, with a total of \(n_V = \N\) visits to the station shared
between \(n_B = \A\) buses. Each BEB has a battery capacity of \(\kappa_b =\) \batsize kWh battery that is required to stay above
an SOC of \(\nu_b =\) \mincharge (\fpeval{\batsize * \minchargeD} kWh). Each bus is assumed to begin the working
day with \(\alpha =\) \fpeval{\acharge*100}\% charge (\fpeval{\acharge * \batsize} kWh). Each bus is also
assumed have a rate of discharge of \(\Delta =\) 30 kW. The penalty method employs a gain of \(D = \Cgain\). A total of \(n_C =\)
\fpeval{\fast + \slow} chargers are utilized where \slow of the chargers are slow charging (\slows kW) and
\fast are fast charging (\fasts kW). As previously introduced, to encourage the SA PAP to utilize the fewest number of
chargers, the value of \(\epsilon_q\) in the objective function is \(\forall q \in \{1,2,..., n_B \}; \epsilon_q = 0\) and \(\forall q \in \{n_B + 1, n_B +
2, ..., n_Q\}; \epsilon_q = 100q\). The SA algorithm utilizes the geometric cooling schedule with an initial temperature of \(T_0
= \tempinit\) with \(\beta_2 = 0.999\), resulting in a total of \(n_M = \tempcnt\) steps. The demand cost is taken over fifteen
minute intervals. Thus let the demand cost be denoted as the peak-15 with the associated symbol \(p_{15}\). A weight
vector of \([3, 3, 2, 1]\) was used to influence the distribution of selecting the new charger, new window, wait, and
slide visit primitives, respectively. The algorithm also assumes a total of \(n_K = \localcnt\) iterations for the local
search at a constant temperature. In total, that results in \fpeval{\localcnt * \tempcnt} configurations
being searched. On average each constant temperature search took an average of \(\quicklocal\) seconds to complete,
resulting in a total runtime of \fpeval{\quicklocal * \tempcnt} seconds.

\ref{sec:sa-heuristic-implementation} introduced the idea of an alternative heuristic implementation for the SA algorithm. To
distinguish the heuristic implementation from the method derived in \ref{sec:sa-generation-mechanisms}, let this implementation
be referred to as ``heuristic'' implementation and the previous as the ``quick'' implementation. Using the same weights for
selecting randomly selecting the primitive generators, the heuristic approach further implemented a weighted
distribution vector of \([3, 1]\) to decide whether to select a slow or fast charger, respectively. In the heuristic
approach, on average the constant temperature search took a total of \(\heuristiclocal\) seconds to complete, resulting in
a total runtime of \fpeval{\heuristiclocal * \tempcnt} seconds. The heuristic generators were expected to be
slightly slower due to its iterative approach.

One of the methods utilized to compare with the SA PAP is the MILP PAP. This framework is the original MILP
implementation of the PAP derived from \cite{qarebagh-2019-optim-sched}. The inputs to the system are the same as those
discussed above. The MILP PAP does not implement the peak-15 in its objective function. In an attempt to compare the
solution of the MILP with the SA output more directly, a similar solve time of 3600 seconds. The MILP was executed
utilizing the Gurobi MILP solver \cite{gurobi-2021-gurob-optim}.

Another heuristic-based optimization strategy, referred to as Qin-Modified, is also employed as a means of comparison
with the results of the SA PAP. The Qin-Modified algorithm is a based on the threshold strategy of
\cite{qin-2016-numer-analy}. The algorithm has been modified slightly to accommodate the case of multiple charger types
without a heuristic search for the best charger type. The heuristic is based on a set of rules that revolve around the
initial charge of the bus at visit \(i\). There are three different thresholds, low (85\%), medium (90\%), and high (95\%).
Buses below the low threshold are prioritized to fast chargers then are allowed to utilize slow chargers if no fast
chargers are available. Buses between the low and medium threshold prioritize slow chargers first and utilize fast
chargers only if no slow chargers are available. Buses above the medium threshold and below high will only be assigned
to slow chargers. Buses above the high threshold will not be charged. Once a bus has been assigned to a charger, it
remains on the charger for the duration of the time it is at the station, or it reaches 95\% charge, whichever comes
first. Note that UTA uses 70\% to decide that a fast charger is required. The previously described simulations were run
on a machine equipped with an AMD Ryzen 9 5900X 12 - Processor (24 core) at 4.95GHz.


\subsection{Results}
\label{sec:sa-results}
The schedules generated by each of the methods is presented in \ref{fig:schedule}. Rows 0-14 represent slow charging
queues and rows 15-29 represent fast charging queues. The symbols represent the initial charge times, and the horizontal
line with the vertical tick signifies the region of time the charger is active. A qualitative comparison between the
different is the perceived lack of organization of the quick SA technique. Although the assignment cost was set in
place, due to the random nature of the queue assignments the simulation was not able to converge to a well-packed
solution. The heuristic approach was able to more successfully able to pack its schedule similarly to the schedule of
the Qin-Modified and MILP PAP.

On the quantitative side of the minimization/packing discussion, the Qin-Modified schedule utilizes one fast and two
slow chargers as can be seen in \ref{subfig:schedule-qin}. The MILP PAP framework generated a schedule that utilizes
three fast charges and four slow chargers as shown in \ref{subfig:schedule-milp}. The heuristic SA strategy created its
schedule with eight slow charger queues and four fast charging queues as shown in \ref{subfig:schedule-heuristic-sa}.
The quick strategy for the SA algorithm created a schedule utilizing fifteen slow and fast chargers as is demonstrated
in \ref{subfig:schedule-quick-sa}. That is to say, the Qin-Modified schedule was able to most effectively minimize the
charger count followed by the MILP PAP, and the heuristic SA, and the work being the quick SA technique. The MILP
produced a schedule with a three charger gap in the fast queues, where the intermediate queues were never used. The
heuristic SA, while possibly being able to move some assignments to a lower charge queue index, did not contain any gaps
of unused queues. The Qin-Modified utilized the fewest chargers overall, but also suffers a lack of optimality in its
packing of the schedule. Thus, none of the methods were able to fully minimize the packing constraint. Note that the SA
algorithm has no guarantee of optimality; therefore, post-processing could be applied to further minimize the charger
indices.

A table of the mean, maximum instantaneous charger use is shown in \ref{tab:charge-count}. The mean is meant to be a
measure of how many slow/fast chargers are being utilized over the time horizon, on average. The maximum and scheduled
rows represent the total amount of queues the system actually utilized in parallel while the scheduled queues represents
the amount of queues a particular schedule calls for. The Qin-Modified only utilized 2 slow chargers, so the mean is
expected to be low (0.788). The quick SA technique utilized every slow charger, but only utilized a mean use of 1.494
chargers with a maximum of only six chargers. This indicates that although all the queues were used, on average only
1.494 queues were required with a peak of six slow queues required to be use in parallel. Thus, with appropriate
packing, only six queues were actually required. Similarly, the MILP on average used 1.877 chargers with a maximum of 4
being used at any given moment, which is the same as the required amount of queues by the schedule. The heuristic SA had
a mean of 1.8 chargers utilized with a maximum of seven chargers utilized, although the total queues required by the
schedule is nine. As stated before, SA will never find the optimum solution; thus, it is to be expected to have
non-optimal assignments.

Similarly, the fast chargers, the Qin-Modified all the queues were utilized but only a mean instantaneous use of 0.234
chargers were used at any given moment with a maximum of two chargers. The MILP calls for a total of three queues, but
only ever has a maximum use of two with a mean of 0.133 chargers utilized at any given moment over the time horizon. The
heuristic SA technique had a required four fast charging queues, but used at most two at a given moment. The average use
over the time horizon was 0.159 fast chargers. That is, both the MILP and heuristic SA schedules could have been further
minimized given the same schedule.

\begin{table}[htbp]
\caption{\label{tab:charge-count}Table of mean and max instantaneous charger usage throughout the time horizon. The schedule row indicates the number of queues required by each schedule.}
\centering
\begin{tabular}{|l|ll|ll|ll|ll|}
\hline
 & MILP &  & Qin-Modified &  & Heuristic &  & Quick & \\[0pt]
\hline
 & Slow & Fast & Slow & Fast & Slow & Fast & Slow & Fast\\[0pt]
Mean & 1.877 & 0.133 & 0.788 & 0.441 & 1.8 & 0.159 & 1.494 & 0.234\\[0pt]
Max & 4 & 2 & 2 & 1 & 7 & 2 & 6 & 2\\[0pt]
Schedule & 4 & 3 & 2 & 1 & 9 & 4 & 15 & 15\\[0pt]
\hline
\end{tabular}
\end{table}

\ref{fig:charge} depicts the initial SOC for each visit throughout the simulation of each framework. \ref{tab:charge}
tabulates the mean, minimum, and maximum SOC upon arrival for each visit. The MILP PAP requires each BEB to stay above
an SOC of 25\% while the quick and heuristic SA approaches heavily penalize a schedule for allowing a BEB to go below the
25\% SOC threshold. The MILP PAP was able to successfully keep the SOC above the threshold (\ref{subfig:milp-charge})
while both SA approaches were not. The SOC of the quick SA approach dropped to a minimum of 29.9 kWh and the heuristic
had a minimum SOC of 6.3 kWh as shown in \ref{tab:charge}. The Qin model allowed the SOC of three BEBs to reach an SOC
of 0\% as shown in \ref{subfig:qin-charge}. The Qin-Modified strategy, being a purely reactive model, is unable to
``sense'' whether a set of routes has a particularly taxing route within time horizon. As such, and in the case of the
example scenario, the BEBs that reached charges of 0\% began with a sequence of short routes, much like the other BEBs.
However, rather than continuing this trend, these sets of routes had one or two longer routes which the Qin-Modified
algorithm was unable to account for. Interestingly, Qin-Modified strategy was able to keep the mean SOC the highest
followed by the quick SA, heuristic SA, and then the MILP, most likely due to the high low, medium, and high thresholds
applied to the algorithm. This measure is useful when viewing with \ref{fig:power} and \ref{fig:energy-usage}.

\begin{table}[htbp]
\caption{\label{tab:charge}Table of mean, min, and max SOC (kWh) for each charging schedule.}
\centering
\begin{tabular}{|l|cccc|}
\hline
 & MILP & Qin-Modifid & Heuristic & Quick\\[0pt]
\hline
Mean & 179.580229533742 & 292.615759538128 & 189.525113828846 & 216.523166855178\\[0pt]
Min & 96.9999999999463 & 0 & 6.3432083 & 29.862568\\[0pt]
Max & 388 & 368.6 & 388 & 388\\[0pt]
\hline
\end{tabular}
\end{table}

\ref{fig:power} depicts the power consumption over the time horizon for each model. As previous stated, the Qin-Modified
schedule had the highest mean SOC over the working day. Referencing \ref{fig:power-usage-milp-qin}, the Qin, while
staying below 1000 kW, is held at that level of power consumption for long periods of time. Similarly, the quick SA,
heuristic SA, and the MILP show that as the average SOC goes down, so does the demand. That is, if a schedule has a
lower average SOC, the required demand (and total energy consumption which will be discussed shortly) is also lower.

\ref{fig:power} is also of interest as it plots the peak power demand over the time horizon. The peaks in descending
order are the quick and heuristic SA are tied at 119.9 kW, the MILP at 1910 kW, and then the Qin at 970 kW. Although the
Qin had the lowest peak, it is again worth noting at this point that the Qin-Modified technique was unable to keep the
SOC above 0\%. Thus, the MILP and quick and heuristic SA are comparable in terms of the demand cost. However, when
viewing the mean power consumption in descending order paints a different picture. The largest mean power demand is the
Qin-Modified at 424.95 kW, quick SA at 257.98 kW, heuristic SA at 198.84 kW, and then the MILP at 177.34 kW. The
Qin-Modified, as shown in \ref{fig:power-usage-milp-qin}, although having the lowest peak, holds the power for a long
duration as previously discussed. This measure directly correlates into the energy consumed by each schedule.

\begin{table}[htbp]
\caption{\label{tab:power}Table of mean and max power demand for each charging schedule.}
\centering
\begin{tabular}{|l|cccc|}
\hline
 & MILP & QM & Heuristic & Quick\\[0pt]
\hline
Mean & 177.34 & 424.95 & 198.84105 & 257.9823\\[0pt]
Max & 1910 & 970 & 1911.9 & 1911.9\\[0pt]
\hline
\end{tabular}
\end{table}

The total energy consumed by each schedule is shown in \ref{fig:energy-usage}. The ordering of most energy consumed to
least is as follows: Qin-Modified, quick SA, heuristic SA, and the MILP PAP. The respective energy consumption for each
technique is: 10198.799 kWh, 6303.1704 kWh, 4797.746 kWh, and 4256.746 kWh. The heuristic SA consuming about 988 kWh
more than the MILP PAP.

\begin{figure}
  \centering
  %~~~~~~~~~~~~~~~~~~~~~~~~~~~~~~~~~~~~~~~~~~~~~~~~~~~~~~~~~~~~~~~~~~~~~~~~~~~~
  % Qin
  \begin{subfigure}[t]{\textwidth}
    \centering
    \includegraphics[width=\textwidth]{img/schedule-quinn}
    \caption{Charging schedule generated by Qin Modified algorithm.}
    \label{subfig:schedule-qin}
  \end{subfigure}

  \hfill

  %%~~~~~~~~~~~~~~~~~~~~~~~~~~~~~~~~~~~~~~~~~~~~~~~~~~~~~~~~~~~~~~~~~~~~~~~~~~~~
  % MILP
  \begin{subfigure}[t]{\textwidth}
    \centering
    \includegraphics[width=\textwidth]{img/schedule-milp}
    \caption{Charging schedule generating by the MILP PAP algorithm.}
    \label{subfig:schedule-milp}
  \end{subfigure}
\end{figure}

\begin{figure} \ContinuedFloat
  \centering

  %%~~~~~~~~~~~~~~~~~~~~~~~~~~~~~~~~~~~~~~~~~~~~~~~~~~~~~~~~~~~~~~~~~~~~~~~~~~~~
  % SA heuristic
  \begin{subfigure}[t]{\textwidth}
    \centering \includegraphics[width=\textwidth]{img/schedule-sa-heuristic}
    \caption{Charging schedule generated by the SA PAP algorithm using the heuristic strategy.}
    \label{subfig:schedule-heuristic-sa}
  \end{subfigure}

  \hfill

  %%~~~~~~~~~~~~~~~~~~~~~~~~~~~~~~~~~~~~~~~~~~~~~~~~~~~~~~~~~~~~~~~~~~~~~~~~~~~~
  % SA quick
  \begin{subfigure}[t]{\textwidth}
    \centering \includegraphics[width=\textwidth]{img/schedule-sa-quick}
    \caption{Charging schedule generated by SA PAP algorithm using the quick strategy.}
    \label{subfig:schedule-quick-sa}
  \end{subfigure}
  \caption{Vairous schedules generated by the different frameworks. The horizonontal line stemming from the nodes ending with a vertical tick indicate the charge duration for that particular visit.}
  \label{fig:schedule}
\end{figure}

\begin{figure}
    %%~~~~~~~~~~~~~~~~~~~~~~~~~~~~~~~~~~~~~~~~~~~~~~~~~~~~~~~~~~~~~~~~~~~~~~~~~~~~
    % Fast
    \begin{subfigure}[t]{\textwidth}
    \centering
        \includegraphics[width=\textwidth]{img/charger-count-fast-milp-qin}
        \caption{Number of fast chargers for Qin and MILP PAP.}
        \label{subfig:fast-charger-usage-milp-qinn}
    \end{subfigure}

    \begin{subfigure}[t]{\textwidth}
    \centering
        \includegraphics[width=\textwidth]{img/charger-count-fast-sa}
        \caption{Number of fast chargers for quick and heuristic SA executions.}
        \label{subfig:fast-charger-usage-sa}
    \end{subfigure}
\end{figure}

\begin{figure}
    %%~~~~~~~~~~~~~~~~~~~~~~~~~~~~~~~~~~~~~~~~~~~~~~~~~~~~~~~~~~~~~~~~~~~~~~~~~~~~
    % Slow
    \begin{subfigure}[t]{\textwidth}
    \centering
        \includegraphics[width=\textwidth]{img/charger-count-slow-milp-qin}
        \caption{Number of slow chargers for Qin and MILP PAP.}
        \label{subfig:slow-charger-usage-milp-qinn}
    \end{subfigure}
    \begin{subfigure}[t]{\textwidth}
    \centering
        \includegraphics[width=\textwidth]{img/charger-count-slow-sa}
        \caption{Number of slow chargers for the quick and heuristic SA executions.}
        \label{subfig:slow-charger-usage-sa}
    \end{subfigure}
\end{figure}

\begin{figure}
  %%~~~~~~~~~~~~~~~~~~~~~~~~~~~~~~~~~~~~~~~~~~~~~~~~~~~~~~~~~~~~~~~~~~~~~~~~~~~~
  % Qin
  \begin{subfigure}[t]{\textwidth}
    \centering
    \includegraphics[width=\textwidth]{img/charge-quinn}
    \caption{Bus charges for the Qin Modified charging schedule.}
    \label{subfig:qin-charge}
  \end{subfigure}
  \hfill
  %%~~~~~~~~~~~~~~~~~~~~~~~~~~~~~~~~~~~~~~~~~~~~~~~~~~~~~~~~~~~~~~~~~~~~~~~~~~~~
  % MILP
  \begin{subfigure}[t]{\textwidth}
    \centering
    \includegraphics[width=\textwidth]{img/charge-milp}
    \caption{The bus charges for the MILP PAP charging schedule.}
    \label{subfig:milp-charge}
  \end{subfigure}
  \hfill
\end{figure}

\begin{figure}\ContinuedFloat
  %%~~~~~~~~~~~~~~~~~~~~~~~~~~~~~~~~~~~~~~~~~~~~~~~~~~~~~~~~~~~~~~~~~~~~~~~~~~~~
  % SA Quick
  \begin{subfigure}[t]{\textwidth}
    \centering
    \includegraphics[width=\textwidth]{img/charge-sa-quick}
    \caption{The bus charges for the quick SA PAP charging schedule.}
    \label{subfig:sa-quick-charge}
  \end{subfigure}
  \hfill
  %%~~~~~~~~~~~~~~~~~~~~~~~~~~~~~~~~~~~~~~~~~~~~~~~~~~~~~~~~~~~~~~~~~~~~~~~~~~~~
  % SA Heuristic
  \begin{subfigure}[t]{\textwidth}
    \centering
    \includegraphics[width=\textwidth]{img/charge-sa-heuristic}
    \caption{The bus charges for the heuristic SA PAP charging schedule.}
    \label{subfig:sa-heuristic-charge}
  \end{subfigure}
  \caption{}
  \label{fig:charge}
\end{figure}

\begin{figure}
  \begin{subfigure}[t]{\textwidth}
    \centering
    \includegraphics[width=\textwidth]{img/power-milp-qin}
    \caption{Amount of power consumed by Qin-Modified and MILP schedules over the time horizon.}
    \label{fig:power-usage-milp-qin}
  \end{subfigure}

  \hfill

  \begin{subfigure}[t]{\textwidth}
    \centering
    \includegraphics[width=\textwidth]{img/power-sa}
    \caption{Amount of power consumed by quick and heuristic SA schedules over the time horizon.}
    \label{fig:power-usage-sa}
  \end{subfigure}
  \caption{}
  \label{fig:power}
\end{figure}

\begin{figure}[htpb]
\centering \includegraphics[width=\textwidth]{img/energy}
    \caption{Total accumulated energy consumed by the Qin-Modified, MILP, quick and heuristic SA schedules throughout the time horizon.}
    \label{fig:energy-usage}
\end{figure}

\section{Conclusion}
\label{sec:sa-conclusion}
This work developed an SA implementation derived from the works of the MILP PAP. The model is designed to reduce the
total number of utilized chargers, minimize the peak energy consumption and the total energy consumed. The problem
description was provided outlining the scenario in which this model is designed for. The optimization problem was then
introduced by describing the components of the objective function and outlining the MILP constraints utilized to ensure
candidate solutions are in the solution space.

An example of the SA PAP algorithm was presented and compared against the MILP PAP and Qin-Modified techniques. The SA
PAP was run utilizing two different neighborhood searching techniques named the quick and heuristic techniques,
respectively. The quick SA's objective was to randomly search a wide neighborhood while the heuristic technique was
designed to incrementally search a neighborhood by randomly selecting a fast or slow charging queue and then stepping
through the queues one at a time. The quick and heuristic have comparable run times at \fpeval{\quicklocal * \tempcnt} seconds and \fpeval{\heuristiclocal * \tempcnt} seconds, respectively, but yielded vastly
different results. The Qin-Modified utilized the fewest amount of chargers followed by the MILP, heuristic SA, then the
quick SA. The assignment cost applied to the objective function had no effect on the results of the quick SA; however,
the heuristic SA was more effective in minimizing the total chargers required. Furthermore, the heuristic SA technique
generated a solution approximating that of the MILP, but was unable to minimize the charger count as efficiently. The
quick SA utilized all the chargers available (i.e. was unable to minimize the charger count).

Both of the SA techniques were unable to keep the SOC above the 25\% SOC threshold with SOC falling to 6.34 kWh for the
heuristic SA and 29.8 kWh. The Qin-Modified had the SOC of three BEBs fall to 0\% SOC. The schedule that consumed the
least amount of energy is the MILP PAP (4256.16 kW) with the heuristic SA coming in second (4797.75 kW). The difference
between the two being about \fpeval{4797.746 - 4256.16} kWh. The peak demands between the heuristic SA, quick
SA, and the MILP were very similar. The MILP had a peak demand of 1910 kW and the quick and heuristic SA had demand
peaks of 1911.9 kW. Overall, the most effective schedule was the MILP; however as SA has no guarantee of optimality, the
heuristic SA was able to generate a schedule that was ``in the ballpark'' of that of the MILP while further taking the
demand cost into consideration.

Further fields of interest are to investigate the performance of the quick and heuristic SA approaches utilizing a denser
set of routes to schedule. It is also of interest to incorporate non-linear battery dynamics to more accurately model
the SOC. Furthermore, ``fuzzifying'' the charge times is of interest to allow flexibility in the initial and final charge
times for each BEB visit.
\vspace{6pt}

\authorcontributions{For research articles with several authors, a short paragraph specifying their individual contributions must be provided. The following statements should be used ``Conceptualization, X.X. and Y.Y.; methodology, X.X.; software, X.X.; validation, X.X., Y.Y. and Z.Z.; formal analysis, X.X.; investigation, X.X.; resources, X.X.; data curation, X.X.; writing---original draft preparation, X.X.; writing---review and editing, X.X.; visualization, X.X.; supervision, X.X.; project administration, X.X.; funding acquisition, Y.Y. All authors have read and agreed to the published version of the manuscript.'', please turn to the  \href{http://img.mdpi.org/data/contributor-role-instruction.pdf}{CRediT taxonomy} for the term explanation. Authorship must be limited to those who have contributed substantially to the work~reported.}

\funding{This research received no external funding.}

\informedconsent{Not applicable.}

\dataavailability{We encourage all authors of articles published in MDPI journals to share their research data. In this section, please provide details regarding where data supporting reported results can be found, including links to publicly archived datasets analyzed or generated during the study. Where no new data were created, or where data is unavailable due to privacy or ethical restrictions, a statement is still required. Suggested Data Availability Statements are available in section ``MDPI Research Data Policies'' at \url{https://www.mdpi.com/ethics}.}

\conflictsofinterest{The authors declare no conflicts of interest.}

\begin{adjustwidth}{-\extralength}{0cm}
\reftitle{References}
\bibliography{c:/msys64/home/1556048963C/Documents/citation-database/lit-ref,c:/msys64/home/1556048963C/Documents/citation-database/lib-ref}
\PublishersNote{}
\end{adjustwidth}
\end{document}